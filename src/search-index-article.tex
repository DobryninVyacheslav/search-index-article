\documentclass[
    twocolumn,
%    hf, %enable header and footer.
]{template/ceurart}

\sloppy
\usepackage{listings}
\lstset{breaklines=true}

\begin{document}
    \copyrightyear{2022}
    \copyrightclause{Copyright for this paper by its authors.
    Use permitted under Creative Commons License Attribution 4.0
    International (CC BY 4.0).}
    \conference{This command is for the conference information}
    \title{Optimization of the structure of a full-text search index using neural networks of the "Transformer" architecture}
    \author[1]{Vyacheslav YU. Dobrynin}
    \author[1]{Roman K. Abramovich}
    \author[1]{Artem D. Gorshkov}
    \author[1]{Alexey V. Platonov}
    \address[1]{ITMO University, Kronverksky Pr. 49, bldg. A, Saint-Petersburg, 197101, Russian Federation}
    \begin{abstract}
        The use of modern language models of the ``Transformer'' architecture in information retrieval significantly
        improves its quality, but negatively affects the performance of the process.
        The paper considers an approach to the formation of a reverse index of a search engine,
        which makes it possible to avoid losses in search speed by precalculating the distances from dictionary
        elements to documents at the indexing stage.
    \end{abstract}
    \begin{keywords}
        Search engine \sep transformer \sep BERT \sep inverted index \sep optimization.
    \end{keywords}
    \maketitle


    \section{Introduction}
    CEUR-WS's article template provides a consistent \LaTeX{} style for
    use across CEUR-WS publications, and incorporates accessibility and
    metadata-extraction functionality.
    This document will explain the
    major features of the document class.
\end{document}
